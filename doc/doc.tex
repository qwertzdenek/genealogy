\documentclass[a4paper, 12pt]{article}

\usepackage{czech}
\usepackage{graphicx}
\usepackage[utf8]{inputenc}
\usepackage{listings}
\usepackage{url}
% \usepackage{a4wide}

\def\CS{$\cal C\kern-.1667em\lower.5ex\hbox{$\cal S$}\kern-.075em $}
\DeclareUrlCommand\url{\def\UrlLeft{<}\def\UrlRight{>} \urlstyle{tt}}

\lstset{language=Java}

\begin{document}
\begin{titlepage}
\includegraphics[bb=0 0 167 96]{fav_cmyk.pdf}
\vfill
\begin{center}
{\huge Umělá inteligence a rozpoznávání}\\[3ex]
{\Large Expertní genealogický systém}
\end{center}
\vfill
\begin{tabbing}
Vypracoval: \hspace{1ex}\=Zdeněk Janeček\kill
Vypracoval: \>Zdeněk \textsc{Janeček}\\[1ex]
Datum:\> \today
\end{tabbing}
\end{titlepage}

\tableofcontents

\section{Zadání}
Vytvořte genealogický expertní systém. Vstupem bude informace databáze
základních informací o osobě [id, jméno, matka, otec, pohlaví, partner].
Program bude umět odpovídat na otázky: Kdo je můj pokrevní příbuzný?
Kolik má moje prateta sestřenic? Jaký je vtah mezi Petrem a Pavlem?\ldots
a další podobné.

\section{Analýza}
% grafové prohledávání, datové struktury

\section{Design}
% Exprertní a klientská třída, UI

\section{Shrnutí}
Při psaní jsem několikrát lehce změnil návrh. V počátku jsem o expertním
systému nevěděl vůbec nic. Ve výsledku jsem použil již ověřené postupy.
Cesta vedla přes zjednodušení celého modelu na jednu expertní třídu.

\end{document}
